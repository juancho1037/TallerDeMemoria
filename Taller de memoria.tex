\documentclass[10pt,letterpaper]{article}
\usepackage[utf8]{inputenc}
\usepackage[T1]{fontenc}
\usepackage[spanish]{babel}
\usepackage{listings}
\usepackage{graphicx}
\graphicspath{ {images/} }
\usepackage{cite}

\usepackage{titlesec}
\usepackage{titletoc}
\usepackage{hyperref}
\usepackage{lipsum}
\usepackage{amsmath, amsthm, amssymb}
\usepackage{ragged2e}
\usepackage{fancyhdr}
\usepackage{subfig}
\usepackage{multirow, array}


\begin{document}
	
	%PORTADA
	\pagestyle{empty}
	
	\begin{figure}[h]
		\centering
		\includegraphics[scale=0.12]{images/escudoUdeA.png}
	\end{figure}
	
	\centering
	
	\textbf{\Large{NOCIONES DE LA MEMORIA DEL COMPUTADOR}}\\
	\vspace{1cm}
	\textbf{\Large{INFORMATICA 2}}\\
	\large
	\vspace{1.4cm}
	\textbf{Juan David Rendon Berrio}\\\vspace{0.1cm}C.C 1037595192 \\\vspace{1cm}      
	
	\textbf{Augusto Salazar}\\
	\vspace{0.2cm}
	\textbf{Jonathan Gómez}\\
	\vspace{0.2cm}
	\Large{Docentes}\\
	\vspace{1cm}
	\vfill
	\large{DEPARTAMENTO DE INGENIERÍA ELECTRÓNICA Y TELECOMUNICACIONES}\\
	\vspace{0.3cm}
	\large{FACULTAD DE INGENIERÍA}\\
	\large{UNIVERSIDAD DE ANTIOQUIA}\\
	\vspace{0.3cm}
	\large{MEDELLIN}\\
	\large{2020-1}\\

\tableofcontents

\newpage
\section{Introducción}

\begin{justify}
	Uno de los grandes temas que tienen que ver con el mundo de la informática, y en lo que todos pensamos ya sea cuando vamos a comprar un computador, o cualquier aparato electrónico; es ¿y cuanta memoria tiene?, pero, en realidad estamos diciendo una palabra sin pensar en su profundidad.\\
	
	\noindent
	La memoria cumple muchas funciones en el mundo de la informática, pero lo primero que se nos viene a la mente para darle significado a esta palabra, es que la memoria es donde uno guarda cosas, cualquier tipo de información, recuerdos, y demás cosas valiosas que uno quiere conservar.\\
	
	\noindent
	En el mundo informático, la memoria cumple varios papeles importantes, en diferentes ámbitos, desde guardar archivos, hasta capacidad de acceso.\\
	
	\noindent
	En este documento se intentará exponer de una manera amigable el significado de la memoria, como funciona, los diferentes tipos que existen, las funciones que desempeña en los diferentes dispositivos electrónicos.
\end{justify}

\newpage
\section{Definición de la memoria del computador} \label{contenido}

\begin{justify}
	Hablando un poco de memoria, es un dispositivo de hardware que retiene datos informáticos de manera permanente o temporal, de acuerdo al tipo, y funcionamiento.\\
	
	\noindent
	Las memorias de computadora, o hablando un poco mas general, de cualquier dispositivo electrónico, proporcionan una de las principales funciones de la computación moderna; la retención o el almacenamiento de información. Es uno de los componentes fundamentales de todas las computadoras, que acoplada a una unidad central de procesamiento (CPU), por su sigla en inglés, “Central processing unit”, implementa lo fundamental del modelo de computadora de Arquitectura de von Neumann, usado desde los años 1940.\\ 
	\cite{definicion}\\
	
	\noindent
	En cuanto al hardware, existen en el mercado muchos tipos de memoria creados y desarrollados para suplir varias necesidades que se fueron presentando en toda la arquitectura computacional.
	
\end{justify}
 
\newpage
\section{Tipos de memoria} \label{conclusion}

\begin{justify}
	En un computador hay varios tipos de memoria, ordenados en jerarquías de velocidad y capacidad.\\
	
	\begin{itemize}
		\item Memoria Cache L1, L2 y L3
		\item Memoria RAM 
		\item Memoria Virtual 
		\item Disco Duro 
			
	\end{itemize}
	Cuanto más bajamos en la lista desde la memoria Cache hacia el disco duro, decrece la velocidad de los distintos tipos de memoria y aumenta su capacidad.\\
	
	\begin{center}
	\textbf{Memoria cache}
	\end{center}
	
	\noindent
	En \underline{informática}, se conoce como \textbf{memoria caché} o memoria de acceso rápido a uno de los recursos con los que cuenta una CPU (Central Processing Unit, o sea, Unidad Central de Procesamiento) para almacenar temporalmente los datos recientemente procesados en un búfer especial, es decir, en una memoria auxiliar.\\
	
	\noindent
	La memoria caché opera de modo similar a la Memoria Principal del CPU, pero con mayor velocidad a pesar de ser de mucho menor tamaño. Su eficacia \textbf{provee al microprocesador de tiempo extra para acceder a los datos más frecuentemente utilizados}, sin tener que rastrearlos a su lugar de origen cada vez que sean necesarios.\\
	
	\noindent
	Así, esta memoria alterna \textbf{se sitúa entre el CPU y la Memoria RAM} (Random Access Memory, o sea, Memoria de Acceso Aleatorio), y provee de un empuje adicional en tiempo y ahorro de recursos al sistema. De allí su nombre, que en inglés significa “escondite”.\\
	
	\noindent
	Existen varios tipos de memoria caché, como los siguientes:\\
	
	\begin{itemize}
		
		\item \textbf{Caché de disco:} Es una porción de memoria RAM asociada a un disco particular, en donde se almacenan los datos de reciente acceso para agilizar su carga.\\
		\item \textbf{Caché de pista:} Similar a la RAM, este tipo de memoria caché sólida empleada por supercomputadores es potente, pero costosa.\\
		
		\item \textbf{Caché de Web:} Se ocupa de almacenar los datos de las páginas Web recientemente visitadas, para agilizar su carga sucesiva y ahorrar ancho de banda. Este tipo de caché a su vez puede funcionar para un solo usuario (privada), varios usuarios a la vez (compartida) o en conjunto para toda la red administrada por un servidor (en pasarela).
		
	\end{itemize}
	

\end{justify}


\noindent
\section{Como se gestiona la memoria en un computador} \label{conclusion}

\begin{justify}
	
	
	
\end{justify}


\noindent
\section{¿Qué hace que una memoria sea más rápida que otra? ¿Por qué esto es importante?} \label{conclusion}

\begin{justify}
	
	
	
\end{justify}

\noindent
\section{Conclusión} \label{conclusion}



\bibliographystyle{apalike}
\bibliography{references}	

\newpage
\noindent








	
	
	
\end{document}