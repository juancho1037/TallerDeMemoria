\documentclass[10pt,letterpaper]{article}
\usepackage[utf8]{inputenc}
\usepackage[T1]{fontenc}
\usepackage[spanish]{babel}
\usepackage{listings}
\usepackage{graphicx}
\graphicspath{ {images/} }
\usepackage{cite}

\usepackage{titlesec}
\usepackage{titletoc}
\usepackage{hyperref}
\usepackage{lipsum}
\usepackage{amsmath, amsthm, amssymb}
\usepackage{ragged2e}
\usepackage{fancyhdr}
\usepackage{subfig}
\usepackage{multirow, array}


\begin{document}
	
	%PORTADA
	\pagestyle{empty}
	
	\begin{figure}[h]
		\centering
		\includegraphics[scale=0.12]{images/escudoUdeA.png}
	\end{figure}
	
	\centering
	
	\textbf{\Large{NOCIONES DE LA MEMORIA DEL COMPUTADOR}}\\
	\vspace{1cm}
	\textbf{\Large{INFORMATICA 2}}\\
	\large
	\vspace{1.4cm}
	\textbf{Juan David Rendon Berrio}\\\vspace{0.1cm}C.C 1037595192 \\\vspace{1cm}      
	
	\textbf{Augusto Salazar}\\
	\vspace{0.2cm}
	\textbf{Jonathan Gómez}\\
	\vspace{0.2cm}
	\Large{Docentes}\\
	\vspace{1cm}
	\vfill
	\large{DEPARTAMENTO DE INGENIERÍA ELECTRÓNICA Y TELECOMUNICACIONES}\\
	\vspace{0.3cm}
	\large{FACULTAD DE INGENIERÍA}\\
	\large{UNIVERSIDAD DE ANTIOQUIA}\\
	\vspace{0.3cm}
	\large{MEDELLIN}\\
	\large{2020-1}\\



\tableofcontents

\newpage
\section{Introducción}

\begin{justify}
	Uno de los grandes temas que tienen que ver con el mundo de la informática, y en lo que todos pensamos ya sea cuando vamos a comprar un computador, o cualquier aparato electrónico; es ¿y cuanta memoria tiene?, pero, en realidad estamos diciendo una palabra sin pensar en su profundidad.\\
	
	\noindent
	La memoria cumple muchas funciones en el mundo de la informática, pero lo primero que se nos viene a la mente para darle significado a esta palabra, es que la memoria es donde uno guarda cosas, cualquier tipo de información, recuerdos, y demás cosas valiosas que uno quiere conservar.\\
	
	\noindent
	En el mundo informático, la memoria cumple varios papeles importantes, en diferentes ámbitos, desde guardar archivos, hasta capacidad de acceso.\\
	
	\noindent
	En este documento se intentará exponer de una manera amigable el significado de la memoria, como funciona, los diferentes tipos que existen, las funciones que desempeña en los diferentes dispositivos electrónicos.
\end{justify}

\newpage
\section{Definición de la memoria del computador} \label{contenido}

Esta sección es para ver qué pasa con los comandos 
que definen texto

El paquete también agrega un comportamiento especial 
a <<estas marcas para hacer citas textuales>> tal como 
lo indican las reglas de la RAE. \cite{dirac}




En la sección de teoremas (\ref{contenido})

\noindent
\section{Conclusión} \label{conclusion}

\bibliographystyle{apalike}
\bibliography{references}	

\newpage
\noindent








	
	
	
\end{document}